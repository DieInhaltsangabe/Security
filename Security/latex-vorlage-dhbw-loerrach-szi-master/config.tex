% Hier müssen die Variablen angepasst werden
% Der Titel der Arbeit, der auf dem Deckblatt angezeigt wird
\def \thesisTitle {Der Titel einer wissenschaftlichen Arbeit kann auch sehr lang werden, aber länger ist nicht unbedingt besser}

% Der Titel der Arbeit, der in der Fußzeile angezeigt wird
\def \thesisFooterTitle {Der Titel einer wissenschaftlichen Arbeit kann auch sehr lang werden, aber länger ist nicht unbedingt besser}

% Typ der Arbeit
\def \thesisType {Seminararbeit}

% Bildungsabschluss
\def \degree {Bachelor of Science}

% Abgabedatum
\def \submissionDate {1. März 2000}

% Studiengang
\def \courseOfStudies {Wirtschaftsinformatik}

% Kurs
\def \course {WWI15B-SE}

% Name des Autors der Arbeit
\def \name {Max Mustermann}

% Name der Ausbildungsfirma
\def \company {Musterfrau AG}

% Ort, an dem Ausbildungsfirma ansässig ist
\def \companyLocation {Freiburg im Breisgau}

% Betreuer der Ausbildungsfirma
\def \corporateAdvisor {Leonie Musterfrau}

% Wissenschaftlicher Betreuer
\def \universityAdvisor {Prof. Dr. Hans Mustermann}

% Ort und Datum für die Ehrenwörtliche Erklärung
\def \declarationLocation {Lörrach}
\def \declarationDate {15. Februar 2000}

% Ort und Datum für die Freigabe der Arbeit
\def \releaseLocation {Freiburg im Breisgau}
\def \releaseDate {01. Februar 2000}

% Name der Bilddatei für das Firmenlogo. Die Datei muss im Ordner images sein.
% Erlaubt sind u.a. folgende Formate: PDF, PNG, JPEG
\def \fileNameLogo {company_logo.pdf}

% Je nach Länge des Titels kann die Schriftgröße angepasst werden
\def \titleFontSize {18}		% Schriftgröße des Titels auf dem Deckblatt
\def \footerFontSize {9}		% Schriftgröße in der Fußzeile

% Wenn es sich um eine Seminararbeit handelt,
% muss der folgende Befehl durch diesen ersetzt werden:
% \seminararbeittrue
\seminararbeitfalse

% Wenn die Arbeit keinen Sperrvermerk hat,
% muss der folgende Befehl durch diesen ersetzt werden:
% \blockingnoticefalse
\blockingnoticetrue

% Die Daten für den Sperrvermerk. Diese müssen natürlich nur geändert werden,
% wenn die Arbeit einen Sperrvermerk hat.
% Datum der Unterschrift des Autors auf dem Sperrvermerk
\def \blockingNoticeAuthorDate {02. Februar 2000}
% Datum der Unterschrift des Unternehmensvertreters auf dem Sperrvermerk
\def \blockingNoticeCompanyDate {03. Februar 2000}
% Adresse und Land des Unternehmens auf dem Sperrvermerk. Die \\ sorgen für einen Zeilenumbruch
\def \companyAdress {Talstraße 1 \\ 79539 Lörrach \\ Deutschland}
% Telefonnummer des Unternehmens auf dem Sperrvermerk.
\def \companyPhone {07621 20710}
% E-Mailadresse des Unternehmens auf dem Sperrvermerk.
\def \companyEmail {info@musterfrau-ag.de}