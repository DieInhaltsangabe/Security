\usepackage{fontspec}
\usepackage{polyglossia}
\usepackage{csquotes}
\setdefaultlanguage[spelling=new, babelshorthands=true]{german}

% Seitenränder
\usepackage[left=3.0cm, right=3.0cm, head=2.5cm, bottom=2.5cm, foot=1cm, includefoot]{geometry}

% Abkürzungsverzeichnis
\usepackage[printonlyused]{acronym}

% Gut formatierte Tabellen
\usepackage{tabulary}

% Positioniert Tabellen und Abbildungen
\usepackage{float}

% Fügt Abschnitte wie den Anhang oder die Kurzfassung dem Inhaltsverzeichnis hinzu
\usepackage{tocbibind}

% Ermöglicht Abbildungen in LaTeX
\usepackage{graphicx}
\graphicspath{ {images/} }

% Definiert die Farben für Tabellen im DHBW-Style
\usepackage[table]{xcolor}
\definecolor{tableHeading}{gray}{0.672}
\definecolor{tableOdd}{gray}{0.945}
\definecolor{tableEven}{gray}{0.859}
\arrayrulecolor{white}

% Schriftart Carlito. Sie ist fast identisch mit Calibri.
% Calibri ist auf Linux und MacOS nicht verfügbar. 
\usepackage{carlito}
\setmainfont{carlito}

% Fußzeile und Kopfzeile
\usepackage{fancyhdr}
\renewcommand{\headrulewidth}{0pt}
\renewcommand{\footrulewidth}{0.4pt}
\renewcommand{\footnotesize}{\fontsize{\footerFontSize}{\footerFontSize}\selectfont}
\fancyhead{}
\fancyfoot{}
\fancyfoot[R]{\hfill \fontsize{\footerFontSize}{\footerFontSize}\selectfont \thepage}
\fancyfoot[C]{
    \parbox{0.8\textwidth}{
    \setlength\topsep{0pt}
    \begin{center}
    \fontsize{\footerFontSize}{\footerFontSize}\selectfont \thesisFooterTitle
    \end{center}
    }
}


% Positioniert die Fußnoten fest am unteren Ende der Seite
\usepackage[bottom]{footmisc}

% Zeilenabstand: 1.5
\usepackage{setspace}
\setstretch{1.5}

% Literaturverzeichnis
\usepackage[natbib=true, backend=biber, style=authoryear, dashed=false]{biblatex}
\DeclareBibliographyAlias{interview}{misc}
\addbibresource{text/bibliography.bib}

% Das Paket hyperref muss als letztes Paket geladen werden. Das Paket setzt Links im PDF-Dokument.
\usepackage[hidelinks, unicode]{hyperref}
\hypersetup{pdftitle = {\thesisTitle}, pdfauthor = {\name}}

% Normale Schriftart für URLs
\renewcommand{\UrlFont}{}

% Variable für das Speichern der Seitenzahl (römisch -> arabisch -> römisch)
\newcounter{pageNumber}

% Nummerierung: 2.1, 2.2 usw.
\renewcommand{\labelenumii}{\theenumii}
\renewcommand{\theenumii}{\theenumi.\arabic{enumii}.}

% Bei Bedarf können die Überschriften geändert werden
\def \declarationHeading{Ehrenwörtliche Erklärung}
\def \thesisSizeHeading{Hinweis zum Umfang der Arbeit}
\def \releaseHeading{Freigabe der Arbeit}
\def \blockingHeading{Sperrvermerk}
\def \abstractHeading{Kurzfassung}
\def \appendixHeading{Anhang}
\def \referenceHeading{Quellenverzeichnis}
\def \acronymHeading{Abkürzungsverzeichnis}

% Befehl für ein einleitendes Zitat
\newcommand{\epigraph}[2]{
    \begin{quote}\begin{quote}
        \begin{center}
            \textit{#1}
        \end{center}
        \hfill #2
    \end{quote}\end{quote}
}