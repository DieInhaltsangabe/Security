\section{Aufsetzten der Umgebung}
Für die Durchführung der Tests wurde die neueste Version von Kali Linux als Live-Distribution verwendet, die mithilfe des Tools Rufus auf einem USB-Stick bootfähig installiert wurde. 
Kali Linux ist eine renommierte Linux-Distribution, die speziell für Penetrationstests und Sicherheitstests entwickelt wurde, und ermöglicht den Zugriff auf aktuelle Sicherheitsfunktionen und Updates.
Im ersten Test wurde auf einem Windows-Laptop ein WLAN-Hotspot eingerichtet, wobei als Sicherheitsstandard WPA2 gewählt wurde. 
Dieser Standard gilt als sicher und wird für drahtlose Netzwerke empfohlen.
Bei der Auswahl der Passwortlänge für die Tests wurde eine Statistik herangezogen, die ergeben hat, dass 49\% der Passwörter eine Länge von bis zu 10 Zeichen aufweisen.\footcite[Vgl]{statistalaenge} 
Aufgrund dieser Erkenntnis wurden für die Tests Passwörter mit einer Länge von genau 10 Zeichen verwendet, um realitätsnahe Szenarien abzubilden.
Insgesamt wurden die Tests unter Verwendung des Kali Linux-Images, das als Live-Distribution auf einem bootfähigen USB-Stick installiert wurde, durchgeführt. 
Dabei wurde ein WLAN-Hotspot mit WPA2 auf einem Windows-Laptop erstellt, während die Auswahl der Passwortlänge auf statistischen Daten basierte. Diese Vorgehensweise gewährleistet eine fundierte und wissenschaftlich solide Durchführung der Tests.
Als Programm für die Brute-Force und Dictionary-Attacken wurde Hashcat verwendet. Hashcat ist ein leistungsstarker Open-Source-Tool zum Cracken von Passwörtern und zur Wiederherstellung von verschlüsselten Hashes. 
Um im folgenden auch noch Rainbow Table Attacken durchzuführen wurde rainbowcrack verwendet. RainbowCrack ist ein leistungsstarkes Tool zur Erzeugung und Verwendung von Rainbow Tables für die Passwortentschlüsselung.
Für das Aufnehmen des Wlan Traffics wird hcxtools verwendet. Dieses ist für hashcat optimiert. 
\section{Durchführung der Test}

 
